% Résumé de lecture

\subsection{In-situ Programmable Switching using rP4: Towards Runtime Data Plane Programmability, \cite{feng_situ_2021}, Feng et al., Nov. 2021}
\gras{Pertinence:} 2
\gras{Sujets:} P4 reconfigurable dynamiquement, PISA

\gras{Problématique:} Comment reconfigurer des commutateurs programmables P4 sans interrompre le service (sans les éteindre/rallumer pendant la configuration)?

\gras{Prérequis:} ECMP \cite{noauthor_equal-cost_2023}, VRF \cite{noauthor_virtual_2021}

\gras{Réponse:} Le modèle PISA empêche la reconfigurabilité, parce que les différentes couches de protocoles sont couplées dans le même parseur, le même pipeline de contrôle et le même désérialiseur; ce qui fait que l'on ne peut pas retirer/ajouter une fonctionnalité sans chambouler l'ensemble du programme. Les auteurs inventent une nouvelle méta-architecture (IPSA) plus adaptée à la reconfigurabilité, proposent un nouveau compilateur adapté à celle-ci (rP4), et ajoutent un nouveau simulateur comportemental pour aller avec (\texttt{ipbm}). Leur architecture vient avec une pénalité de surface et d'énergie d'environ 15\%, après essai sur FPGA (sans pipelinage).

\gras{Résumé:}

I - Introduction et proposition de valeur

Les auteurs insistent sur la nécéssité d'ajouter la reprogrammabilité dynamique pour faires des tests de configurations, pouvoir installer puis retirer des fonctions de télémétrie INT dont on a besoin que temporairement, ajouter/enlever des fonctions de calcul-sur-réseau temporairement aussi, et expérimenter avec de nouveaux protocoles. La reconfigurabilité dynamique va aussi devenir essentielle au fur et à mesure qu'on avance vers des réseaux autonomes. Pour l'atteindre, ils réinventent toute la chaîne d'outils P4 depuis le début: méta-architecture, compilateur, simulateur.

II - L'architecture IPSA

Dans PISA, il y a un parseur, un pipeline de contrôle et un sérialiseur globaux pour toute la pile de protocole. L'implémentation de chaque couche réseau est fortement couplée aux autres; ce qui rend complexe toute mise-à-jour incrémentale. Leur architecture IPSA remédie à ce problème en séparant le traitement des couches, chacune placée dans un bloc dédié appelé TSP (Templated Stage Processor). Chaque TCP comprend le parseur, les tables et la logique de contrôle (pas de désérialiseurs). La pipeline d'un TSP à l'autre peut être flexible: les TSP peuvent être connectés linéairement, l'un après l'autre, ou un bus d'interconnexions (qu'ils appelent TM sans en définir l'acronyme) capable de sauter un ou plusieurs TSP; et qui connecte aussi le dernier TSP au désérialiseur global. Le TM permet d'insérer/enlever un TSP du pipeline sans reconfigurer toute la chaîne.

Concernant les tables, les TSP ne font que pointer vers des blocs de SRAM/TCAM allouée d'un bassin global auxquel les TSP sont connectés au moyen d'un crossbar.

III - Le rP4

Le rP4 est un nouveau langage basé sur P4 pour programmer selon le paradigme IPSA. Ils ont conçu toute une panoplie d'outils pour développer avec:
\begin{itemize}
	\item{\texttt{rp4fc} (rP4 front-compiler): transpile le P4 normal en syntaxe rP4, génère aussi l'API pour le plan de contrôle (?)}
	\item{\texttt{rp4bc} (back-compiler): donne un JSON adapté à la cible, qui indique sur quels blocs les TSP doivent être placés. Il peut aussi générer des m-à-j incrémentales, ajout/suppression de TSP dans le chemin. Leurs algorithmes priorisent la latence au détriment de la surface.}
\end{itemize}

IV - Implémentation et évaluation

Ils testent 3 cas, créés à partir d'une architecture de routage Ethernet/IP normale, avec une variation à chaque fois: C1) ajout de la fonction ECMP ; C2) ajout du SRv6 ; C3) insertion d'un renifleur de paquets. Ils codent eux-même une implémentation VHDL de PISA, puis de IPSA (ce qui fait que leur comparaison est plus ou moins biaisée mais passons). Ils programment aussi leur compilateur et un utilitaire en ligne de commandes pour charger/décharger les modules.

Résultats: Au final les bandes-passantes sont divisées par trois avec IPSA; mais c'est surtout parce qu'ils ont pas encore pipeliné leur implémentation pour atteindre les une instruction par cycle. Les temps de compilation/flashage des mises-à-jour incrémentales sont divisés par 5-10 par rapport à une recompilation/reconfiguration intégrales PISA. De même pour la compilation et mise-à-jour de modèles comportementaux avec leur simulateur \texttt{ipbm} par rapport à \texttt{bmv2}.

V - Synthèse matérielle et résultats

Ils implémentent leur modèle sur Alveo 280. Leur implémentation prend \~15\% plus de LUT, 60\% plus de DFF (!), même si les DFF sont pas si important que ça.

La consommation d'énergie est supérieure à partir de 5 TSP, c'est-à-dire pour toute implémentation réaliste, mais les auteurs considèrent que pour un design avec énormément d'étages IPSA devrait diminuer la consommation totale puisqu'il permet d'éteindre les coeurs non-utilisés.

VI - Littérature similaire

D'autres solutions visent le même but que la leur, mais avec des limites: parfois la solution incrémentale n'est pas transposable en matériel, parfois ils faut quand même recompiler tout le programme pour une màj incrémentale.

\gras{Avis:} Un framework très intéressant; j'aime beaucoup leur idée de réduire le couplage entre les étages dans le parseur et le contrôle. Leur méthode de màj incrémentale me parrait aussi très utilisable. Ils ont tout implémenté sur FPGA; il serait intéressant de voir comment on peut partitionner leur architecture sur un système hétérogène.

\clearpage
