% Résumé de lecture

\subsection{The P4$\rightarrow$NetFPGA Workflow for line-rate packet processing, \cite{ibanez_p4_2019}, Ibanez, Brebner, McKeown, Zilberman, Fév. 2019, FPGA-2019}
\gras{Pertinence:} 2
\gras{Sujets:} P4->HDL, Outils

\gras{Problématique:} Créer un flot de développement P4 pour la recherche et l'enseignement, qui utilise du matériel réel mais n'exige pas de connaissances en HDL.

\gras{Prérequis:} (Optionnel) SNMP \cite{noauthor_simple_2023} \cite{digitalocean_introduction_nodate}

\gras{Résumé:} 

I - Introduction

Généralités sur le langage P4 tel qu'il était en 2019. Utilité d'avoir un environnement de développement qui abstraie les détails du RTL dans l'enseignement et de la réseautique.

II - Structure du P4

Exposés sur la structure du P4: structure des en-têtes, structure des tables, étages de contrôle, déparseur.

Définition des \texttt{externs}, qui permet à la partie P4 d'une architecture d'interagir avec le reste du matériel. Typiquement utilisés pour implémenter des fonctions \italique{stateful}, auxquelles le P4 n'est pas encore bien adapté.

Différence entre les métadonnées simples, et les métadonnées standard (\texttt{standard\_metadata}):
\begin{itemize}
	\item les métadonnées sont définies au choix par le programmeur P4
	\item les métadonnées standard sont fixées dans l'architecture, et servent à interagir avec celle-ci. Par exemple, on choisit généralement le port de sortie d'un paquet en réglant \texttt{standard\_metadata.egress\_port}, et on jette un paquet \texttt{standard\_metadata.drop}
\end{itemize}

III - Présentation et historique de SDNet

SDNet est l'ancien compilateur P4$\fleche$HDL de Xilinx, à l'origine conçu pour le langage PX. Le PX est un langage de traitement de paquets similaire mais antérieur au P4. Ce langage était propriétaire et a éventuellement été remplacé par le P4. SDNet prend en entrée un programme $P4_{16}$, et génère 1 - Le design Verilog à synthétiser 2 - Un environnement de vérification SystemVerilog 3 - Des pilotes logiciel pour le plan de contrôle, en langage C.

IV - Proposition de valeur

Il a existé plusieurs itérations de la carte NetFPGA depuis 2004, mais toutes ont eu en commun d'exiger de leurs utilisateurs qu'ils maîtrisent le développement RTL. L'auteur propose plutôt de créer un flot direct P4$\fleche$HDL, qui permette de créer un prototype fonctionnel sans pour autant faire de développement matériel. Les usages cibles sont l'enseignement de la réseautique et le développement de prototypes de recherche. Cela sous-entend que le système proposé n'est pas fait pour aller en production!

V - Description de l'infrastructure de développement

La structure du P4 favorise surtout les transformations \an{stateless}. Pour faciliter le prototypage \an{stateful}, ils ont implémenté des \texttt{externs} basiques de lecture/écritures de registres. Il y a aussi un \texttt{extern} pour l'horodatage IEEE 1588. Le compilateur P4 génère un bloc de traitement intégré au design plus large \texttt{SimpleSUMESwitch}. On ne peut pas facilement passer à une autre architecture.

Il est possible de simuler les designs à niveaux avant de passer à l'implémentation matérielle:
\begin{enumerate}
	\item Envoi et vérification de paquets vers un bloc de traitement P4 seul, sans le reste des modules RTL qui forment la NIC ou le commutateur complet.
	\item Simulation de la NIC ou du commutateur entier, incluant le bloc de traitement P4 et le autres modules RTL ex. DMA, files, SerDes...
\end{enumerate}

Ils ont aussi créé un API python, construite par-dessus les pilotes C, pour expérimenter plus facilement avec le plan de contrôle.

VI - Cas d'utilisation en recherche et enseignement

Ils listent les cas remarquables d'utilisation de leur flot de développement en recherche: une implémentation de INT, de gestion de trafic proactive (adaptée automatiquement en fonction de la topologie du réseau et pas en réaction à des problèmes), d'équilibrage de trafic, et enfin de vérification du plan de données. Ils ont aussi pu enseigner un cours de réseau à Stanford à partir de cette plateforme, à des étudiants sans connaissance des FPGA.

VII - Améliorations futures

S'ils avaient continué à développer la plateforme, ils se seraient penchés sur l'intégration de P4-Runtime et de systèmes d'exploitations pour commutateurs utilisant P4-Runtime, entre autres ONOS. Ils auraient aussi facilité la création de nouvelles architectures.

VIII - Comparaisons aux solution P4-à-HDL existantes

Les autres plateformes de synthèse P4 pour FPGA utilisent un langage HLS intermédiaire: C++ ou encore Bluespec. Ces HLS ne donnent pourtant pas d'abstractions qui soient réellement utile au traitement de paquets. Globalement, les systèmes concurrents sont soit propriétaires, soit des travaux de recherche qui ne présentent pas comme des outils de développement pratiques.

IX-X - Conclusion: La NetFPGA remplit bien son rôle d'intermédiaire entre l'émulation logicielle et les ASICs chers.

\gras{Avis:} Les auteurs comparent leur solution aux flots P4->HLS->FPGA existants (ils citent Jeferson \cite{santiago_da_silva_p4-compatible_2018}). Leur solution se démarque par sa facilité d'utilisation: elle est intégrée aux outils et passe directement du P4 au FPGA sans passer par l'intermédiaire d'un HLS.
Cependant, nous sommes aujourd'hui 4 ans après la publication du papier, et l'outil n'a pas été maintenu depuis ce temps-là... le compilateur SDNet est devenu obsolète et non-téléchargeable, et les développeurs de son successeur VitisNetP4 n'ont pas l'intention de le faire fonctionner sur la famille Virtex-7, qu'utilise la NetFPGA-SUME: \url{https://support.xilinx.com/s/question/0D74U000007t5w0SAA/}.

On est obligés de ressortir nos anciens compilateurs pour ne pas que nos cartes à 7000\$ ne deviennent pas des presse-papiers.

\clearpage
