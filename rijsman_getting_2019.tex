% Résumé de lecture

\subsection{Getting started with P4, \cite{rijsman_getting_2019}, Rijsman, 2019}
\gras{Pertinence:} 3 (tutoriel pour monter une VM P4 et envoyer quelques paquets tests)
\gras{Sujets:} langage P4, environnements de développement

\gras{Problématique:} Quelles sont les étapes pour configurer un environnement P4 simple, monter un commutateur P4 basique, et le faire fonctionner sur le BMv2?

\gras{Prérequis:} Maîtrise de VirtualBox/VMWare/autre \cite{noauthor_oracle_2004} (l'auteur fait tout dans une boîte AWS, mais c'est quand même plus pratique avec une VM); protobuf; interfaces virtuelles Linux

\gras{Réponse:} Liste des étapes ci-dessous:

\gras{Résumé:}

\begin{itemize}
	\item Monter une VM Ubuntu 20.04.
	\item Ouvrir le SSH dessus (la manière la plus simple de le faire est de régler une IP statique sur la VM avec \texttt{netplan}, configurer le réseau de la VM en \emph{bridge}).
	\item Installer le million de dépendances \texttt{apt} - sauf \texttt{texlive-full}, je ne vois pas pourquoi ils suggèrent ce paquet plus lourd à lui seul que tout les autres, ça fonctionne très bien sans.
	\item Compiler et installer protobuf
	\item Compiler et installer BMv2
	\item Taper le code P4 basé sur \texttt{<core.p4>} et \texttt{<v1model.p4>}.
	\item Compiler avec \texttt{p4c} vers le BMv2. Ils sort un \texttt{.json} lisible par le simulateur comportemental.
	\item Ajouter une paire d'interface virtuelles \texttt{veth} à la boîte Linux.
	\item Lancer le commutateur \texttt{simple\_switch}, lui passer les interface virtuelles en option
	\item Lancer le plan de contrôle \texttt{simple\_switch\_CLI}, remplir la table de routage IP.
	\item Lancer des renifleurs de paquets sur les interfaces virtuelles, par exemple \texttt{tcpdump}.
	\item Envoyer des paquets test avec \texttt{scapy}. Selon l'IP de destination choisie ils apparaissent dans les fenêtres \texttt{tcpdump} en sortie.
\end{itemize}

\gras{Avis:} Le seul tutoriel de référence trouvable qui explique comment monter l'environnement de A à Z, sans utiliser la VM préfabriquée du répo officiel. Il marche du premier coup. Par contre il n'explique pas \emph{pourquoi} on a besoin de telle ou telle dépendance. Ça reste à explorer. Les outils sont déjà matures et fonctionne déjà bien. Reste à voir comment ça se passe quand on veut compiler vers des cibles matérielles.
\clearpage
